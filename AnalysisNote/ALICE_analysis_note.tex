\documentclass[ALICE,manyauthors]{ALICE_analysis_notes}
%\documentclass[ALICE,manyauthors]{ALICE_scientific_notes}
%
%\newcommand{\jpsi}{\rm J/$\psi$}
%\newcommand{\psip}{$\psi^\prime$}
%\newcommand{\jpsiDY}{\rm J/$\psi$\,/\,DY}
\newcommand{\dd}{\mathrm{d}}
%\newcommand{\chic}{$\chi_{\rm c}$}
%\newcommand{\ezdc}{$E_{\rm ZDC}$}
%\newcommand{\red}{\textcolor{red}}
\newcommand{\blue}{\textcolor{blue}}
\newcommand{\slfrac}[2]{\left.#1\right/#2}
\usepackage{rotating}
\usepackage{listings}
\usepackage{multirow}
\usepackage{tabularx}
\usepackage{comment}
\usepackage{xcolor}
\usepackage{listings}
\usepackage{siunitx}
\usepackage[backend=bibtex,style=numeric,sorting=none]{biblatex} % Use the bibtex backend with the authoryear citation sty le (which resembles APA)
\addbibresource{ALICE_analysis_note} 
\usepackage{hyperref}

\DeclareSIUnit\clight{\text{\ensuremath{c}}}
%
\begin{document}%
%%%%%%%%%%%%% ptdr definitions %%%%%%%%%%%%%%%%%%%%%
%
%%%%%%%%%%%%%%%  Title page %%%%%%%%%%%%%%%%%%%%%%%%
%
\begin{titlepage}
%
\PHnumber{ALICE-ANA-2014-xxx} 
\PHdate{\today}
%
%%% Put your own title + short title here:
\title{First measurement of $\phi$-pair production in pp collisions}
\ShortTitle{First measurement of $\phi$-pair production in pp collisions}   % appears on right page headers
%
\author{Nicola Rubini$^{1}$}
\author{
1. University and INFN Bologna\\
}
\author{Email: Nicola.Rubini@cern.ch}
%
\ShortAuthor{ALICE Analysis Note 2012}      % appears on left page headers, do not change
%
\begin{abstract}
%The measurement of the φ meson at mid-rapidity in minimum-bias pp collisions at s = 13 TeV is described in this note. The results include the pT spectrum, the integrated yield dN/dy, the mean transverse momentum ⟨pT⟩, and ratios to the yields of other hadrons. A multiplicity-dependent analysis will be performed separately.
In this Analysis Note we present the new Analysis Technique for the measurement of the $\phi$-meson pair yield. The $\phi$ meson is reconstructed via its $\phi(1020)\to$K$^{+}$K$^{-}$ decay channel (B.R. 49.2$\pm$0.5\%). Results will be presented about Invariant Mass histograms, inclusive spectra, conditional spectra and a differentiation in multiplicity of the measurement is performed.
\end{abstract}
\end{titlepage}
%
\tableofcontents
\newpage
%
%	SECTIONS
%
\setcounter{secnumdepth}{0}
\section{Introduction}
\label{sec:Introduction}
This note describes the measurement using the ALICE detector of $\phi$ mesons and $\phi$-meson pairs produced in minimum-bias pp collisions at $\sqrt{s}=$\SI{7}{\tera\electronvolt}. In this analysis the $\phi$ mesons are reconstructed using their decay in a Kaon couple of opposite sign, which has a branching ratio of $(49.2\%\pm 0.5\%)$ \cite[PDG]. The Invariant Mass technique is used for the reconstruction of the mesons, both for inclusive $\phi$ mesons and $\phi$-meson pairs. For the latter a 2D generalisation is required, this will be the topic of chapter \ref{sec:2D_Technique}.

\begin{comment}
 In this analysis, decays of φ mesons to charged kaons are reconstructed. The yield of φ mesons is extracted from KK invariant-mass distributions as a function of transverse momentum. The φ spectrum is integrated to obtain a measurement of the total dN/dy, and the mean transverse momentum ⟨pT⟩ is extracted from the spectrum.
The analysis procedure described herein closely follows the procedure used to analyze φ mesons in Pb–Pb collisions at √sNN = 2.76 TeV in 2010 data [1]. The basic invariant-mass analysis is performed many times, with the method varied each time (i.e., using different PID cuts, different combinatorial backgrounds, and many other variations). In order to distinguish between the many different varia- tions of this analysis, it will be useful to introduce shorthand notations, which will be written in “type- writer” font like this. Multiple shorthand notations may be concatenated using underscores like this: tpc2s mix0 fr3 pol2 yh1. Each shorthand notation will be written in blue when first introduced.
\end{comment}

\subsection{Analysis Summary}
Title of Note: First measurement of $\phi$-pair production in minimum-bias pp collisions at $\sqrt{s}=$\SI{7}{\tera\electronvolt}.\\
Objective: This note documents the analysis of $\phi$-meson pairs in minimum-bias pp collisions at $\sqrt{s}=$\SI{7}{\tera\electronvolt} which are intended for inclusion in a forthcoming paper.\\
Primary Author: Nicola Rubini, University of Bologna\\
TWiki Address: TBD
\[...\]
\begin{comment}
This section is a list of information required by the Physics Board.
Title of Note: Measurement of φ Mesons in Minimum-Bias pp Collisions at s = 13 TeV
Objective: This note documents the analysis of φ mesons in minimum-bias pp collisions at which are intended for inclusion in a forthcoming paper.
Primary Author: Anders G. Knospe, The University of Houston
TWiki Address: https://twiki.cern.ch/twiki/bin/view/ALICE/PWGLFResonancesPhiPP13TeV
s = 13 TeV,
AliRoot/AliPhysics versions: AliRoot::v5-08-09-1 + AliPhysics::vAN-20160502-1
Data Samples Used: real data: LHC15f, ESDs, pass 2
Analysis General and Specific Selections: Standard Physics Selection, |vz| < 10 cm, pileup rejection Detectors: ITS, TPC, TOF
Description of cuts: standard ITS/TPC track cuts for 2011, see Section 4.1.
Simulations: LHC15g3a3, LHC15g3c3, and LHC16d3; ESDs
Discussion of Efficiencies, Corrections, Etc.: see Section 7.1
Normalization: see Section 7.4
Discussion of Uncertainties: see Sections 8, 9.2, and 9.3.
Relevant Presentations:
\end{comment}
\subsection{Source Code}
The latest version of the Analysis Task used to fetch Data and Monte Carlo simulations can be seen \href{https://github.com/alisw/AliPhysics/tree/master/PWGLF/RESONANCES/extra}{here}. The files of interested for the presented analysis are:\\
\href{https://github.com/alisw/AliPhysics/blob/master/PWGLF/RESONANCES/extra/AddAnalysisTaskPhiCount.C}{\texttt{AddAnalysisTaskPhiCount.C}}.\\
\href{https://github.com/alisw/AliPhysics/blob/master/PWGLF/RESONANCES/extra/AliAnalysisTaskPhiCount.cxx}{\texttt{AliAnalysisTaskPhiCount.cxx}}.\\
\href{https://github.com/alisw/AliPhysics/blob/master/PWGLF/RESONANCES/extra/AliAnalysisTaskPhiCount.h}{\texttt{AliAnalysisTaskPhiCount.h}}.\\
\indent All the analysis code, comprehensive of plot generation macros, can be found \href{https://github.com/Nikolajal/AliAnalysisPhiCount}{here}, using the latest version of the package \href{https://github.com/Nikolajal/AliAnalysisUtility.git}{\texttt{AliAnalysisUtility}}
\setcounter{secnumdepth}{1}
\newpage
\section{Dataset and event selection}
\label{sec:Dataset_and_event_selection}

A total of about $XXX$ events are selected and used for the analysis. A brief summary of their properties, together with their correspondent Monte Carlo production, are liste in Table \ref{tab:datasetsummary}. 

\begin{table}
\center
\begin{tabularx}{\textwidth}{c|cccc}
Type		&Dataset		&N$_{runs}$		&N$_{events}$		&AliRoot\\
\hline

\end{tabularx}
\caption{Datasets and Monte Carlo production used in the analysis}
\label{tab:datasetsummary}
\end{table}

\subsection{Vertex selection}
The first requirement for the candidate event is a proper vertex. This means:
\begin{enumerate}
\item The vertex has to be reconstructed by the SPD
\item If the track reconstructed vertex is not available, the SPD vertex is taken
\item If the track reconstructed vertex is available, the z coordinate of the two are compared and the event is discarded if the two are more than \SI{0.5}{\centi\meter} apart.
\item If the accepted vertex absolute value of the z coordinate is more than \SI{10}{\centi\meter}, the event is discarded.
\end{enumerate}
that the vertex is required to be reconstructed by the SPD

\subsection{Multiplicity}
The information about multiplicity by the V0M detector must be present.

\subsection{Pile-up}
The vent is discarded if flagged as pile.up form the SPD in multiplicity bins


\subsection{TODO}
\begin{enumerate}
\item Update the trigger type, trigger efficiency
\item Put complete runlist in appendix
\end{enumerate}

\begin{comment}

TO BE PUT IN THE APPENDIX

\begin{table}
\center
\begin{tabularx}{\textwidth}{c|X}
Dataset		&Runlist\\
\hline
LHC10b		&117222, 117220, 117116, 117112, 117099, 117092, 117063, 117060, 117059, 117053, 117052, 117050, 117048, 116645, 116643, 116574, 116571, 116562, 116403, 116402, 116288, 116102, 116081, 116079, 115414, 115401, 115399, 115393, 115345, 115335, 115328, 115322, 115318, 115310, 115193, 115186, 114931, 114930, 114924, 114918, 114798, 114786\\
					\hline
LHC10c		&121040, 121039, 120829, 120825, 120824, 120823, 120822, 120821, 120758, 120750, 120741, 120671, 120617, 120616, 120505, 120503, 120244, 120079, 120076, 120073, 120072, 120069, 120067, 119862, 119859, 119856, 119853, 119849, 119846, 119845, 119844, 119842, 119841, 118561, 118560, 118558, 118556, 118518, 118506\\
					\hline
LHC10d		&126158 , 126097, 126090, 126088, 126082, 126081, 126078, 126073, 126008, 126007, 126004, 125855, 125851, 125850, 125849, 125848, 125847, 125844, 125843, 125842, 125633, 125632, 125630, 125628, 125296, 125134, 125101, 125100, 125097, 125085, 125083, 125023, 122375, 122374\\
					\hline
LHC10e		&130850, 130848, 130847, 130844, 130842, 130840, 130834, 130799, 130798, 130795, 130793, 130704, 130696, 130628, 130623, 130621, 130620, 130609, 130608, 130524, 130520, 130519, 130517, 130481, 130480, 130479, 130375, 130178, 130172, 130168, 130158, 130157, 130149, 129983, 129966, 129962, 129961, 129960, 129744, 129742, 129738, 129736, 129735, 129734, 129729, 129726, 129725, 129723, 129666, 129659, 129653, 129652, 129651, 129650, 129647, 129641, 129639, 129599, 129587, 129586, 129540, 129536, 129528, 129527, 129525, 129524, 129523, 129521, 129520, 129514, 129513, 129512, 129042, 128913, 128855, 128853, 128850, 128843, 128836, 128835, 128834, 128833, 128824, 128823, 128820, 128819, 128778, 128777, 128678, 128677, 128621, 128615, 128611, 128609, 128605, 128582, 128506, 128505, 128504, 128503, 128498, 128495, 128494, 128486\\			
					\hline	
LHC17p		&282343, 282342, 282341, 282340, 282314, 282313, 282312, 282309, 282307, 282306, 282305, 282304, 282303, 282302, 282247, 282230, 282229, 282227, 282224, 282206, 282189, 282147, 282146, 282127, 282126, 282125, 282123, 282122, 282120, 282119, 282118, 282099, 282098, 282078, 282051, 282050, 282031, 282030, 282025, 282021, 282016, 282008\\			
					\hline
\end{tabularx}
\caption{Runlists used for the analysis with corresponding datasets}
\label{tab:dataset}
\end{table}




\end{comment}
\begin{comment}
TODO:	----------------------------
1. Add PID Histograms for each variation cut
2. PID shortcuts
------------------------------------------
\end{comment}

\section{Track Selection and Particle Identification}
\label{sec:Trackslelection}
As was mentioned above, the $\phi$ meson reconstruction was performed using the Invariant mass technique on its decay product K$^{+}$K$^{-}$, which have a branching ratio of 49.2(5)\% \cite{PDG}. The main target of our identification effort is then primary charged Kaons.\\
\subsection{Track selection}
To this end, we resort to the \href{https://twiki.cern.ch/twiki/bin/viewauth/ALICE/AliDPGtoolsFilteringCuts#Run_flag_1000_AddTrackCutsLHC10b}{DPG Track Filterbit 5} that encapsules the \texttt{GetStandardITSTPCTrackCuts2010()},which applies a number of selections, listed below:
\begin{enumerate}
\item A minimum number of rows crossed in the TPC ($N_{cr,TPC}$ $\geq 70$)
\item A maximum $\chi^2$ per cluster in the TPC ($\chi^2_{TPC} < 4$)
\item Reject kink daughters
\item Require ITS refits
\item Require TPC refits
\item Minimum number of clusters in SPD: 1 ($\texttt{AliESDtrackCuts::SetClusterRequirementITS(kSPD, kAny)}$)
\item DCAxy $<$ 0.0182 + 0.0350$/p_{T}^{1.01}$ (7-$\sigma$ cut)
\item A maximum $\chi^2$ per TPC-constrained Global ($\chi^2_{CGI} < 36$)
\item DCAz $<$ \SI{2}{\centi \meter}
\item $\texttt{AliESDtrackCuts::SetDCAToVertex2D(kFALSE)}$
\item $\texttt{AliESDtrackCuts::SetRequireSigmaToVertex(kFALSE)}$
\item A maximum $\chi^2$ per cluster in the ITS ($\chi^2_{ITS} < 36$)
\end{enumerate}
In addition to these selections we add a cut in $\eta$, $p_{T}$ and rapidity to enhance the precision of the measurement:
\begin{enumerate}
\item p$_{T}$ of Kaon candidate over \SI{0.15}{\giga\electronvolt} (p$_{T} \geq$ \SI{0.15}{\giga\electronvolt})
\item $\eta$ of Kaon candidate in range [-0.8;0.8] ($|\eta| < 0.8$)
\item Reconstructed $\phi$ candidate in rapidity range [-0.5;0.5] ($|\text{y}| < 0.5$)
\end{enumerate}
\subsection{PID Selection}
Once the primary tracks are selected, we proceed to the particle identification using the TPC and TOF detectors, respectively measuring the energy loss and the time of flight of the particle. The selection is made using the $\sigma_K$ of the detector, representing the probability of the particle of being a Kaon based on the measurement of the specific detector. The selections used in the analysis are:
\begin{enumerate}
\item If the track does not match a TOF hit, a $|\sigma_K^{\text{TPC}}| < 3.0$ selection is performed
\item If the track matches a TOF hit, a $|\sigma_K^{\text{TPC}}| < 5.0$ selection is performed, combined with a $|\sigma_K^{\text{TOF}}| < 3.0$ selection (TOF veto)
\end{enumerate}
Given the presence of issues in the TPC reconstruction for low momenta particles, the TPC selection is widened to $|\sigma_K^{\text{TPC}}| < 7.0$ for tracks having a 
p$_{T} \leq$ \SI{0.28}{\giga\electronvolt}. This PID selection will not be concerned by variations made to evaluate the systematic uncertainty.
\subsection{Uncertainty evaluation}
To evaluate the uncertainties associated to the selections the analysis is run multiple times varying the PID selection values of $\sigma_K$ by $10\%$:
\begin{enumerate}
\item $|\sigma_K^{\text{TPC}}| < 3.0$ $\qquad$ $|\sigma_K^{\text{TPC}}| < 5.0$ with $|\sigma_K^{\text{TOF}}| < 3.0$ veto $\qquad$ (Standard)
\item $|\sigma_K^{\text{TPC}}| < 3.3$ $\qquad$ $|\sigma_K^{\text{TPC}}| < 5.0$ with $|\sigma_K^{\text{TOF}}| < 3.0$ veto 
\item $|\sigma_K^{\text{TPC}}| < 2.7$ $\qquad$ $|\sigma_K^{\text{TPC}}| < 5.0$ with $|\sigma_K^{\text{TOF}}| < 3.0$ veto 
\item $|\sigma_K^{\text{TPC}}| < 3.0$ $\qquad$ $|\sigma_K^{\text{TPC}}| < 5.5$ with $|\sigma_K^{\text{TOF}}| < 3.0$ veto
\item $|\sigma_K^{\text{TPC}}| < 3.0$ $\qquad$ $|\sigma_K^{\text{TPC}}| < 4.5$ with $|\sigma_K^{\text{TOF}}| < 3.0$ veto
\item $|\sigma_K^{\text{TPC}}| < 3.0$ $\qquad$ $|\sigma_K^{\text{TPC}}| < 5.0$ with $|\sigma_K^{\text{TOF}}| < 3.3$ veto
\item $|\sigma_K^{\text{TPC}}| < 3.0$ $\qquad$ $|\sigma_K^{\text{TPC}}| < 5.0$ with $|\sigma_K^{\text{TOF}}| < 2.7$ veto
\end{enumerate}
For the tracking efficiency, a $4\%$ uncertainty is taken from \cite{PrevPub}

\section{Signal Extraction}
\label{sec:SignalExtraction}
The analysis makes use of Invariant Mass histograms built using different p$_T$ bins of eligible candidates. The list of bins used in the analysis are listed in Table \ref{tab:PTbins}.

\begin{table}
\center
\begin{tabular}{ccc|ccc||ccc}
\multicolumn{6}{c}{1D Analysis} 					&\multicolumn{3}{c}{2D Analysis}\\
Bin	&Min		&Max	&Bin		&Min		&Max	&Bin		&Min		&Max\\
\hline
1	&0.4		&...		&8		&0.4		&...		&1		&0.4		&...\\
2	&0.4		&...		&9		&0.4		&...		&2		&0.4		&...\\
3	&0.4		&...		&10		&0.4		&...		&3		&0.4		&...\\
4	&0.4		&...		&11		&0.4		&...		&4		&0.4		&...\\
5	&0.4		&...		&12		&0.4		&...		&5		&0.4		&...\\
6	&0.4		&...		&13		&0.4		&...		&6		&0.4		&...\\
7	&0.4		&...		&14		&0.4		&...		&7		&0.4		&...\\
\end{tabular}
\label{tab:PTbins}
\caption{The p$_{T}$ bins used in the analysis, all values are in \SI{}{\giga \electronvolt \per \clight}}
\end{table}

\subsection{Extraction of $\phi$ meson}
The Signal extraction of $\phi$ meson yields is performed using the Invariant mass distribution of un-like signed charged Kaons pairs from the same event. In this analysis we do not make use of the background subtraction method to maintain consistency with the pair analysis.
\subsubsection{Background estimation}
The background estimation is made simultaneously to the Signal Extraction using a Fit procedure with a function that accounts for the two components. For the part regarding the background a Chebychev polynomial of the 4$^{th}$ degree is used. 
\subsubsection{Signal Peak Fit}
The signal peak fit is performed with a Vogtian function.

\subsection{Extraction of $\phi$-meson pair}
The Signal extraction of $\phi$ meson yields is performed using the 2-Dimensional Invariant mass distribution of un-like signed charged Kaons pairs from the same event. The 2-Dimensional distribution combines Kaon pairs ($\phi$-meson candidates) from within the same event, having the first pair on the x-axis and the second candidate on the y-axis. Given the choice of the pair order is in principle arbitrary, a symmetrisation process is performed where the distribution is built with two entries, weighted $\frac{1}{2}$ each, where the candidates are swapped in their position.\\
\indent When selecting the candidates further selections for background rejection are implemented:
\begin{enumerate}
\item Only events with more than 2 K$^{+}$ and 2K$^{-}$ are used
\item For each $\phi$-meson candidate, a check on the Kaons used to build it is made: no $\phi$-meson candidate pair should share Kaons.
\end{enumerate}
\subsubsection{Background estimation}
The $\phi$-meson pair analysis presents three kinds of Background one needs to discriminate against:
\begin{enumerate}
\item x-candidate: Combinatorial Background	$\qquad$ 		y-candidate: Combinatorial Background
\item x-candidate: True $\phi$-meson decay	$\qquad$ 		y-candidate: Combinatorial Background
\item x-candidate: Combinatorial Background	$\qquad$ 		y-candidate: True $\phi$-meson decay
\end{enumerate}
In order to 
\subsubsection{Signal Peak Fit}

\subsection{Uncertainty Evaluation}
As a check on the goodness of this choice the process is repeated with a Chebychev polynomial of 3$^{rd}$ and 5$^{th}$ degree.


\begin{comment}
\end{comment}
\section{Dataset and event selection}
\label{sec:Dataset_and_event_selection}

A total of about $XXX$ events are selected and used for the analysis. A brief summary of their properties, together with their correspondent Monte Carlo production, are liste in Table \ref{tab:datasetsummary}. 

\begin{table}
\center
\begin{tabularx}{\textwidth}{c|cccc}
Type		&Dataset		&N$_{runs}$		&N$_{events}$		&AliRoot\\
\hline

\end{tabularx}
\caption{Datasets and Monte Carlo production used in the analysis}
\label{tab:datasetsummary}
\end{table}

\subsection{Vertex selection}
The first requirement for the candidate event is a proper vertex. This means:
\begin{enumerate}
\item The vertex has to be reconstructed by the SPD
\item If the track reconstructed vertex is not available, the SPD vertex is taken
\item If the track reconstructed vertex is available, the z coordinate of the two are compared and the event is discarded if the two are more than \SI{0.5}{\centi\meter} apart.
\item If the accepted vertex absolute value of the z coordinate is more than \SI{10}{\centi\meter}, the event is discarded.
\end{enumerate}
that the vertex is required to be reconstructed by the SPD

\subsection{Multiplicity}
The information about multiplicity by the V0M detector must be present.

\subsection{Pile-up}
The vent is discarded if flagged as pile.up form the SPD in multiplicity bins


\subsection{TODO}
\begin{enumerate}
\item Update the trigger type, trigger efficiency
\item Put complete runlist in appendix
\end{enumerate}

\begin{comment}

TO BE PUT IN THE APPENDIX

\begin{table}
\center
\begin{tabularx}{\textwidth}{c|X}
Dataset		&Runlist\\
\hline
LHC10b		&117222, 117220, 117116, 117112, 117099, 117092, 117063, 117060, 117059, 117053, 117052, 117050, 117048, 116645, 116643, 116574, 116571, 116562, 116403, 116402, 116288, 116102, 116081, 116079, 115414, 115401, 115399, 115393, 115345, 115335, 115328, 115322, 115318, 115310, 115193, 115186, 114931, 114930, 114924, 114918, 114798, 114786\\
					\hline
LHC10c		&121040, 121039, 120829, 120825, 120824, 120823, 120822, 120821, 120758, 120750, 120741, 120671, 120617, 120616, 120505, 120503, 120244, 120079, 120076, 120073, 120072, 120069, 120067, 119862, 119859, 119856, 119853, 119849, 119846, 119845, 119844, 119842, 119841, 118561, 118560, 118558, 118556, 118518, 118506\\
					\hline
LHC10d		&126158 , 126097, 126090, 126088, 126082, 126081, 126078, 126073, 126008, 126007, 126004, 125855, 125851, 125850, 125849, 125848, 125847, 125844, 125843, 125842, 125633, 125632, 125630, 125628, 125296, 125134, 125101, 125100, 125097, 125085, 125083, 125023, 122375, 122374\\
					\hline
LHC10e		&130850, 130848, 130847, 130844, 130842, 130840, 130834, 130799, 130798, 130795, 130793, 130704, 130696, 130628, 130623, 130621, 130620, 130609, 130608, 130524, 130520, 130519, 130517, 130481, 130480, 130479, 130375, 130178, 130172, 130168, 130158, 130157, 130149, 129983, 129966, 129962, 129961, 129960, 129744, 129742, 129738, 129736, 129735, 129734, 129729, 129726, 129725, 129723, 129666, 129659, 129653, 129652, 129651, 129650, 129647, 129641, 129639, 129599, 129587, 129586, 129540, 129536, 129528, 129527, 129525, 129524, 129523, 129521, 129520, 129514, 129513, 129512, 129042, 128913, 128855, 128853, 128850, 128843, 128836, 128835, 128834, 128833, 128824, 128823, 128820, 128819, 128778, 128777, 128678, 128677, 128621, 128615, 128611, 128609, 128605, 128582, 128506, 128505, 128504, 128503, 128498, 128495, 128494, 128486\\			
					\hline	
LHC17p		&282343, 282342, 282341, 282340, 282314, 282313, 282312, 282309, 282307, 282306, 282305, 282304, 282303, 282302, 282247, 282230, 282229, 282227, 282224, 282206, 282189, 282147, 282146, 282127, 282126, 282125, 282123, 282122, 282120, 282119, 282118, 282099, 282098, 282078, 282051, 282050, 282031, 282030, 282025, 282021, 282016, 282008\\			
					\hline
\end{tabularx}
\caption{Runlists used for the analysis with corresponding datasets}
\label{tab:dataset}
\end{table}




\end{comment}
\section{Dataset and event selection}
\label{sec:Dataset_and_event_selection}

A total of about $XXX$ events are selected and used for the analysis. A brief summary of their properties, together with their correspondent Monte Carlo production, are liste in Table \ref{tab:datasetsummary}. 

\begin{table}
\center
\begin{tabularx}{\textwidth}{c|cccc}
Type		&Dataset		&N$_{runs}$		&N$_{events}$		&AliRoot\\
\hline

\end{tabularx}
\caption{Datasets and Monte Carlo production used in the analysis}
\label{tab:datasetsummary}
\end{table}

\subsection{Vertex selection}
The first requirement for the candidate event is a proper vertex. This means:
\begin{enumerate}
\item The vertex has to be reconstructed by the SPD
\item If the track reconstructed vertex is not available, the SPD vertex is taken
\item If the track reconstructed vertex is available, the z coordinate of the two are compared and the event is discarded if the two are more than \SI{0.5}{\centi\meter} apart.
\item If the accepted vertex absolute value of the z coordinate is more than \SI{10}{\centi\meter}, the event is discarded.
\end{enumerate}
that the vertex is required to be reconstructed by the SPD

\subsection{Multiplicity}
The information about multiplicity by the V0M detector must be present.

\subsection{Pile-up}
The vent is discarded if flagged as pile.up form the SPD in multiplicity bins


\subsection{TODO}
\begin{enumerate}
\item Update the trigger type, trigger efficiency
\item Put complete runlist in appendix
\end{enumerate}

\begin{comment}

TO BE PUT IN THE APPENDIX

\begin{table}
\center
\begin{tabularx}{\textwidth}{c|X}
Dataset		&Runlist\\
\hline
LHC10b		&117222, 117220, 117116, 117112, 117099, 117092, 117063, 117060, 117059, 117053, 117052, 117050, 117048, 116645, 116643, 116574, 116571, 116562, 116403, 116402, 116288, 116102, 116081, 116079, 115414, 115401, 115399, 115393, 115345, 115335, 115328, 115322, 115318, 115310, 115193, 115186, 114931, 114930, 114924, 114918, 114798, 114786\\
					\hline
LHC10c		&121040, 121039, 120829, 120825, 120824, 120823, 120822, 120821, 120758, 120750, 120741, 120671, 120617, 120616, 120505, 120503, 120244, 120079, 120076, 120073, 120072, 120069, 120067, 119862, 119859, 119856, 119853, 119849, 119846, 119845, 119844, 119842, 119841, 118561, 118560, 118558, 118556, 118518, 118506\\
					\hline
LHC10d		&126158 , 126097, 126090, 126088, 126082, 126081, 126078, 126073, 126008, 126007, 126004, 125855, 125851, 125850, 125849, 125848, 125847, 125844, 125843, 125842, 125633, 125632, 125630, 125628, 125296, 125134, 125101, 125100, 125097, 125085, 125083, 125023, 122375, 122374\\
					\hline
LHC10e		&130850, 130848, 130847, 130844, 130842, 130840, 130834, 130799, 130798, 130795, 130793, 130704, 130696, 130628, 130623, 130621, 130620, 130609, 130608, 130524, 130520, 130519, 130517, 130481, 130480, 130479, 130375, 130178, 130172, 130168, 130158, 130157, 130149, 129983, 129966, 129962, 129961, 129960, 129744, 129742, 129738, 129736, 129735, 129734, 129729, 129726, 129725, 129723, 129666, 129659, 129653, 129652, 129651, 129650, 129647, 129641, 129639, 129599, 129587, 129586, 129540, 129536, 129528, 129527, 129525, 129524, 129523, 129521, 129520, 129514, 129513, 129512, 129042, 128913, 128855, 128853, 128850, 128843, 128836, 128835, 128834, 128833, 128824, 128823, 128820, 128819, 128778, 128777, 128678, 128677, 128621, 128615, 128611, 128609, 128605, 128582, 128506, 128505, 128504, 128503, 128498, 128495, 128494, 128486\\			
					\hline	
LHC17p		&282343, 282342, 282341, 282340, 282314, 282313, 282312, 282309, 282307, 282306, 282305, 282304, 282303, 282302, 282247, 282230, 282229, 282227, 282224, 282206, 282189, 282147, 282146, 282127, 282126, 282125, 282123, 282122, 282120, 282119, 282118, 282099, 282098, 282078, 282051, 282050, 282031, 282030, 282025, 282021, 282016, 282008\\			
					\hline
\end{tabularx}
\caption{Runlists used for the analysis with corresponding datasets}
\label{tab:dataset}
\end{table}




\end{comment}
\section{Dataset and event selection}
\label{sec:Dataset_and_event_selection}

A total of about $XXX$ events are selected and used for the analysis. A brief summary of their properties, together with their correspondent Monte Carlo production, are liste in Table \ref{tab:datasetsummary}. 

\begin{table}
\center
\begin{tabularx}{\textwidth}{c|cccc}
Type		&Dataset		&N$_{runs}$		&N$_{events}$		&AliRoot\\
\hline

\end{tabularx}
\caption{Datasets and Monte Carlo production used in the analysis}
\label{tab:datasetsummary}
\end{table}

\subsection{Vertex selection}
The first requirement for the candidate event is a proper vertex. This means:
\begin{enumerate}
\item The vertex has to be reconstructed by the SPD
\item If the track reconstructed vertex is not available, the SPD vertex is taken
\item If the track reconstructed vertex is available, the z coordinate of the two are compared and the event is discarded if the two are more than \SI{0.5}{\centi\meter} apart.
\item If the accepted vertex absolute value of the z coordinate is more than \SI{10}{\centi\meter}, the event is discarded.
\end{enumerate}
that the vertex is required to be reconstructed by the SPD

\subsection{Multiplicity}
The information about multiplicity by the V0M detector must be present.

\subsection{Pile-up}
The vent is discarded if flagged as pile.up form the SPD in multiplicity bins


\subsection{TODO}
\begin{enumerate}
\item Update the trigger type, trigger efficiency
\item Put complete runlist in appendix
\end{enumerate}

\begin{comment}

TO BE PUT IN THE APPENDIX

\begin{table}
\center
\begin{tabularx}{\textwidth}{c|X}
Dataset		&Runlist\\
\hline
LHC10b		&117222, 117220, 117116, 117112, 117099, 117092, 117063, 117060, 117059, 117053, 117052, 117050, 117048, 116645, 116643, 116574, 116571, 116562, 116403, 116402, 116288, 116102, 116081, 116079, 115414, 115401, 115399, 115393, 115345, 115335, 115328, 115322, 115318, 115310, 115193, 115186, 114931, 114930, 114924, 114918, 114798, 114786\\
					\hline
LHC10c		&121040, 121039, 120829, 120825, 120824, 120823, 120822, 120821, 120758, 120750, 120741, 120671, 120617, 120616, 120505, 120503, 120244, 120079, 120076, 120073, 120072, 120069, 120067, 119862, 119859, 119856, 119853, 119849, 119846, 119845, 119844, 119842, 119841, 118561, 118560, 118558, 118556, 118518, 118506\\
					\hline
LHC10d		&126158 , 126097, 126090, 126088, 126082, 126081, 126078, 126073, 126008, 126007, 126004, 125855, 125851, 125850, 125849, 125848, 125847, 125844, 125843, 125842, 125633, 125632, 125630, 125628, 125296, 125134, 125101, 125100, 125097, 125085, 125083, 125023, 122375, 122374\\
					\hline
LHC10e		&130850, 130848, 130847, 130844, 130842, 130840, 130834, 130799, 130798, 130795, 130793, 130704, 130696, 130628, 130623, 130621, 130620, 130609, 130608, 130524, 130520, 130519, 130517, 130481, 130480, 130479, 130375, 130178, 130172, 130168, 130158, 130157, 130149, 129983, 129966, 129962, 129961, 129960, 129744, 129742, 129738, 129736, 129735, 129734, 129729, 129726, 129725, 129723, 129666, 129659, 129653, 129652, 129651, 129650, 129647, 129641, 129639, 129599, 129587, 129586, 129540, 129536, 129528, 129527, 129525, 129524, 129523, 129521, 129520, 129514, 129513, 129512, 129042, 128913, 128855, 128853, 128850, 128843, 128836, 128835, 128834, 128833, 128824, 128823, 128820, 128819, 128778, 128777, 128678, 128677, 128621, 128615, 128611, 128609, 128605, 128582, 128506, 128505, 128504, 128503, 128498, 128495, 128494, 128486\\			
					\hline	
LHC17p		&282343, 282342, 282341, 282340, 282314, 282313, 282312, 282309, 282307, 282306, 282305, 282304, 282303, 282302, 282247, 282230, 282229, 282227, 282224, 282206, 282189, 282147, 282146, 282127, 282126, 282125, 282123, 282122, 282120, 282119, 282118, 282099, 282098, 282078, 282051, 282050, 282031, 282030, 282025, 282021, 282016, 282008\\			
					\hline
\end{tabularx}
\caption{Runlists used for the analysis with corresponding datasets}
\label{tab:dataset}
\end{table}




\end{comment}
\section{Dataset and event selection}
\label{sec:Dataset_and_event_selection}

A total of about $XXX$ events are selected and used for the analysis. A brief summary of their properties, together with their correspondent Monte Carlo production, are liste in Table \ref{tab:datasetsummary}. 

\begin{table}
\center
\begin{tabularx}{\textwidth}{c|cccc}
Type		&Dataset		&N$_{runs}$		&N$_{events}$		&AliRoot\\
\hline

\end{tabularx}
\caption{Datasets and Monte Carlo production used in the analysis}
\label{tab:datasetsummary}
\end{table}

\subsection{Vertex selection}
The first requirement for the candidate event is a proper vertex. This means:
\begin{enumerate}
\item The vertex has to be reconstructed by the SPD
\item If the track reconstructed vertex is not available, the SPD vertex is taken
\item If the track reconstructed vertex is available, the z coordinate of the two are compared and the event is discarded if the two are more than \SI{0.5}{\centi\meter} apart.
\item If the accepted vertex absolute value of the z coordinate is more than \SI{10}{\centi\meter}, the event is discarded.
\end{enumerate}
that the vertex is required to be reconstructed by the SPD

\subsection{Multiplicity}
The information about multiplicity by the V0M detector must be present.

\subsection{Pile-up}
The vent is discarded if flagged as pile.up form the SPD in multiplicity bins


\subsection{TODO}
\begin{enumerate}
\item Update the trigger type, trigger efficiency
\item Put complete runlist in appendix
\end{enumerate}

\begin{comment}

TO BE PUT IN THE APPENDIX

\begin{table}
\center
\begin{tabularx}{\textwidth}{c|X}
Dataset		&Runlist\\
\hline
LHC10b		&117222, 117220, 117116, 117112, 117099, 117092, 117063, 117060, 117059, 117053, 117052, 117050, 117048, 116645, 116643, 116574, 116571, 116562, 116403, 116402, 116288, 116102, 116081, 116079, 115414, 115401, 115399, 115393, 115345, 115335, 115328, 115322, 115318, 115310, 115193, 115186, 114931, 114930, 114924, 114918, 114798, 114786\\
					\hline
LHC10c		&121040, 121039, 120829, 120825, 120824, 120823, 120822, 120821, 120758, 120750, 120741, 120671, 120617, 120616, 120505, 120503, 120244, 120079, 120076, 120073, 120072, 120069, 120067, 119862, 119859, 119856, 119853, 119849, 119846, 119845, 119844, 119842, 119841, 118561, 118560, 118558, 118556, 118518, 118506\\
					\hline
LHC10d		&126158 , 126097, 126090, 126088, 126082, 126081, 126078, 126073, 126008, 126007, 126004, 125855, 125851, 125850, 125849, 125848, 125847, 125844, 125843, 125842, 125633, 125632, 125630, 125628, 125296, 125134, 125101, 125100, 125097, 125085, 125083, 125023, 122375, 122374\\
					\hline
LHC10e		&130850, 130848, 130847, 130844, 130842, 130840, 130834, 130799, 130798, 130795, 130793, 130704, 130696, 130628, 130623, 130621, 130620, 130609, 130608, 130524, 130520, 130519, 130517, 130481, 130480, 130479, 130375, 130178, 130172, 130168, 130158, 130157, 130149, 129983, 129966, 129962, 129961, 129960, 129744, 129742, 129738, 129736, 129735, 129734, 129729, 129726, 129725, 129723, 129666, 129659, 129653, 129652, 129651, 129650, 129647, 129641, 129639, 129599, 129587, 129586, 129540, 129536, 129528, 129527, 129525, 129524, 129523, 129521, 129520, 129514, 129513, 129512, 129042, 128913, 128855, 128853, 128850, 128843, 128836, 128835, 128834, 128833, 128824, 128823, 128820, 128819, 128778, 128777, 128678, 128677, 128621, 128615, 128611, 128609, 128605, 128582, 128506, 128505, 128504, 128503, 128498, 128495, 128494, 128486\\			
					\hline	
LHC17p		&282343, 282342, 282341, 282340, 282314, 282313, 282312, 282309, 282307, 282306, 282305, 282304, 282303, 282302, 282247, 282230, 282229, 282227, 282224, 282206, 282189, 282147, 282146, 282127, 282126, 282125, 282123, 282122, 282120, 282119, 282118, 282099, 282098, 282078, 282051, 282050, 282031, 282030, 282025, 282021, 282016, 282008\\			
					\hline
\end{tabularx}
\caption{Runlists used for the analysis with corresponding datasets}
\label{tab:dataset}
\end{table}




\end{comment}
\section{Systematic uncertainties evaluation}
\label{sec:Systematics}

Systematic uncertainties are evaluated running the analysis multiple times changing different values of parameters. The variations for each source of uncertainty are described in the dedicated sections or listed here. 


A detailed summary of the Systematic uncertainties used are given below:

\paragraph{Global Tracking Efficiency}
The global tracking efficiency expresses the difference in TPC-ITS matching probability between data and simulations. It is an uncertainty, the results do not require a correction for this contribution. In this work we do not evaluate it, we borrow it from \cite{PrevPubMult}, with a value of 8\%.

\paragraph{Analysis Cuts}
[...]

\paragraph{PID Selection}
[...] look-up \ref{sec:Trackslelection}

\paragraph{Signal Extraction}
[...] look-up \ref{sec:SignalExtraction}

\paragraph{Material Budget}
[...]

\paragraph{Hadronic Interaction}
[...]




\begin{comment}
337 PID selections (Epid ) :
338 Three different PID selection criteria are used for both K∗ and φ. For K∗, 2σ is default and it is varied
339 as 1.5σ and 2.5σ and for φ , 3σ is default and it is varied as 2.0σ and 2.5σ . Due to low statistics we are
340 not able to do the systematic checks for analysis cuts in the different multiplicity classes. Therefore the
341 minimum bias uncertainties are assigned to all the multiplicity classes.


342 Signal extractions :
343 To estimate the systematic uncertainties due to signal extractions, we have varied combinatorial back-
344 ground normalization range, residual background function, fitting range, width of the distribution and
345 signal counting. The details of these variations is given below :
346
347 1. Combinatorial ackground normalization (Enorm) : The different normalization regions are se-
348 lected for K∗ and φ. In case of K∗, the background histogram is normalized in the region of invariant
349 mass between 1.1 to 1.15 GeV/c2 as a default range. For the systematic study the variation in the nor-
350 malization range on the right hand side of the signal is 1.0, 1.1, 1.1-1.2 GeV/c2 and on left hand side
351 of the signal 0.7-0.8 GeV/c2. In case of φ, the default range is 1.045 to 1.05 GeV/c2 and the systematic
352 variation is 0.995-1.0, 1.035-1.045 GeV/c2 .
353 2. Residual background function (Erb) : The default function used for the estimation of residual
Multiplicity dependence of resonances production 23
354 background if 2nd order polynomial for both K∗ and φ. The systematic uncertainty due residual back-
355 ground is estimated by comparing the 3rd order polynomial residual background with the default one.
356 3. Fitting range (E f it r ) : For φ -meson, fitting ranges are varied by 0.005 from the default fitting range
357 (DFR) i.e. [(DFR - 0.005, DFR + 0.005), (DFR + 0.005, DFR - 0.005)]. In case of K∗ the default range is
358 0.78 - 1.05 and for systematic study the range variations are 0.77-1.03, 0.76-1.06, 0.79-1.04.
359 4. Free width (Efw) : For φ, the width is fixed to the PDG value and extracted the resolution as
360 free parameter from the data in default settings.. Then we fixed these pT dependent resolutions for
361 systematic check. In case of K∗ width is kept free for systematic variation.
362 5. Signal count (Esc) : We have used method to calculate the signal : bin counting and function
363 integration. In case of K∗, bin counting method is used for systematic variation, however for φ function
364 integration is used for systematic variations. The systematic uncertainty due signal count is estimated
365 by comparing the default method with the other method.


366 Material budget (Emb) : The material budget systematic uncertainty is estimated by using the pT
367 dependent material budget uncertainty of single K and π. In simulation (official PYTHIA), for a given
368 pT , first the decay daughters of K∗ and φ mesons are identified. Then the daughters are assigned an
369 uncertainty corresponding to that pT bin. These uncertainties are linearly added (assigned for that of
370 K∗ or φ) and a TProfile is filled for each pT bin of K∗ and φ-meson. The bin content of the TProfile
371 which gives the average value is taken as the systematic uncertainty.


372 Hadronic interaction (Ehi ) :
373 The same procedure as described in case of material budget is followed to estimate the systematic
374 uncertainty due to the hadronic interaction for K∗ and φ -meson.
375 Total Systematic Uncertainty (Et ot al ) The total systematic uncertainty is the quadrature sum of all
376 sources.
E =􏰃E2 +E2 +E2 +E2 +E2 +E2 +E2 +E2 +E2 +E2 (22) tot sc rb fw ftr sc anacut pid gtrk mb hd
377 For the systematic study we repeat the measurement by varying one parameter at a time. A Barlow [6]
378 check has been performed for each measurement to verify whether it is due to a systematic effect or a
379 statistical fluctuation.



381 Barlow check
382 Let each measurement be indicated by (yi ± σi) and the ficentral valuefi (default measurement) by
383 (yc ±σc), one can define ∆σi (eqn. 23).
∆σi =􏰃|σi2−σc2| (23)
384 Then we calculate ni = ∆yi/∆σi, where ∆yi = |yc −yi|. If ni ≤ 1.0 then the effects are due to the statistical
385 fluctuation otherwise there is a systematic effect which means that the two measurements are not
386 compatible within the statistical errors.
387 The measurements which passed the fiBarlowfi check (ni > 1) are used to determine the systematic
388 uncertainty. For measurements N > 2, the systematic uncertainty has been determined as the RMS
389 (eqn. 24) of the available measurements otherwise the same is the absolute difference between them.
   
24 ALICE Analysis Note 2015
 􏰅12
δySyst. = N ∑(yi −y ̄) (24)
i
390 Here N is the total number of available measurements including yc and y ̄ is the average of value of the
391 measurements. Whenever only two measurements were available and did not pass “Barlow” check,
392 zero systematic uncertainty has been assigned to the value.

\end{comment}
\setcounter{secnumdepth}{0}
%\section{Dataset and event selection}
\label{sec:Dataset_and_event_selection}

A total of about $XXX$ events are selected and used for the analysis. A brief summary of their properties, together with their correspondent Monte Carlo production, are liste in Table \ref{tab:datasetsummary}. 

\begin{table}
\center
\begin{tabularx}{\textwidth}{c|cccc}
Type		&Dataset		&N$_{runs}$		&N$_{events}$		&AliRoot\\
\hline

\end{tabularx}
\caption{Datasets and Monte Carlo production used in the analysis}
\label{tab:datasetsummary}
\end{table}

\subsection{Vertex selection}
The first requirement for the candidate event is a proper vertex. This means:
\begin{enumerate}
\item The vertex has to be reconstructed by the SPD
\item If the track reconstructed vertex is not available, the SPD vertex is taken
\item If the track reconstructed vertex is available, the z coordinate of the two are compared and the event is discarded if the two are more than \SI{0.5}{\centi\meter} apart.
\item If the accepted vertex absolute value of the z coordinate is more than \SI{10}{\centi\meter}, the event is discarded.
\end{enumerate}
that the vertex is required to be reconstructed by the SPD

\subsection{Multiplicity}
The information about multiplicity by the V0M detector must be present.

\subsection{Pile-up}
The vent is discarded if flagged as pile.up form the SPD in multiplicity bins


\subsection{TODO}
\begin{enumerate}
\item Update the trigger type, trigger efficiency
\item Put complete runlist in appendix
\end{enumerate}

\begin{comment}

TO BE PUT IN THE APPENDIX

\begin{table}
\center
\begin{tabularx}{\textwidth}{c|X}
Dataset		&Runlist\\
\hline
LHC10b		&117222, 117220, 117116, 117112, 117099, 117092, 117063, 117060, 117059, 117053, 117052, 117050, 117048, 116645, 116643, 116574, 116571, 116562, 116403, 116402, 116288, 116102, 116081, 116079, 115414, 115401, 115399, 115393, 115345, 115335, 115328, 115322, 115318, 115310, 115193, 115186, 114931, 114930, 114924, 114918, 114798, 114786\\
					\hline
LHC10c		&121040, 121039, 120829, 120825, 120824, 120823, 120822, 120821, 120758, 120750, 120741, 120671, 120617, 120616, 120505, 120503, 120244, 120079, 120076, 120073, 120072, 120069, 120067, 119862, 119859, 119856, 119853, 119849, 119846, 119845, 119844, 119842, 119841, 118561, 118560, 118558, 118556, 118518, 118506\\
					\hline
LHC10d		&126158 , 126097, 126090, 126088, 126082, 126081, 126078, 126073, 126008, 126007, 126004, 125855, 125851, 125850, 125849, 125848, 125847, 125844, 125843, 125842, 125633, 125632, 125630, 125628, 125296, 125134, 125101, 125100, 125097, 125085, 125083, 125023, 122375, 122374\\
					\hline
LHC10e		&130850, 130848, 130847, 130844, 130842, 130840, 130834, 130799, 130798, 130795, 130793, 130704, 130696, 130628, 130623, 130621, 130620, 130609, 130608, 130524, 130520, 130519, 130517, 130481, 130480, 130479, 130375, 130178, 130172, 130168, 130158, 130157, 130149, 129983, 129966, 129962, 129961, 129960, 129744, 129742, 129738, 129736, 129735, 129734, 129729, 129726, 129725, 129723, 129666, 129659, 129653, 129652, 129651, 129650, 129647, 129641, 129639, 129599, 129587, 129586, 129540, 129536, 129528, 129527, 129525, 129524, 129523, 129521, 129520, 129514, 129513, 129512, 129042, 128913, 128855, 128853, 128850, 128843, 128836, 128835, 128834, 128833, 128824, 128823, 128820, 128819, 128778, 128777, 128678, 128677, 128621, 128615, 128611, 128609, 128605, 128582, 128506, 128505, 128504, 128503, 128498, 128495, 128494, 128486\\			
					\hline	
LHC17p		&282343, 282342, 282341, 282340, 282314, 282313, 282312, 282309, 282307, 282306, 282305, 282304, 282303, 282302, 282247, 282230, 282229, 282227, 282224, 282206, 282189, 282147, 282146, 282127, 282126, 282125, 282123, 282122, 282120, 282119, 282118, 282099, 282098, 282078, 282051, 282050, 282031, 282030, 282025, 282021, 282016, 282008\\			
					\hline
\end{tabularx}
\caption{Runlists used for the analysis with corresponding datasets}
\label{tab:dataset}
\end{table}




\end{comment}
\setcounter{secnumdepth}{1}
%
\printbibliography
%
\end{document}
