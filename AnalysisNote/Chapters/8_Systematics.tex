\section{Systematic uncertainties evaluation}
\label{sec:Systematics}

Systematic uncertainties are evaluated running the analysis multiple times changing different values of parameters. The variations for each source of uncertainty are described in the dedicated sections or listed here. 


A detailed summary of the Systematic uncertainties used are given below:

\paragraph{Global Tracking Efficiency}
The global tracking efficiency expresses the difference in TPC-ITS matching probability between data and simulations. It is an uncertainty, the results do not require a correction for this contribution. In this work we do not evaluate it, we borrow it from \cite{PrevPubMult}, with a value of 8\%.

\paragraph{Analysis Cuts}
[...]

\paragraph{PID Selection}
[...] look-up \ref{sec:Trackslelection}

\paragraph{Signal Extraction}
[...] look-up \ref{sec:SignalExtraction}

\paragraph{Material Budget}
[...]

\paragraph{Hadronic Interaction}
[...]




\begin{comment}
337 PID selections (Epid ) :
338 Three different PID selection criteria are used for both K∗ and φ. For K∗, 2σ is default and it is varied
339 as 1.5σ and 2.5σ and for φ , 3σ is default and it is varied as 2.0σ and 2.5σ . Due to low statistics we are
340 not able to do the systematic checks for analysis cuts in the different multiplicity classes. Therefore the
341 minimum bias uncertainties are assigned to all the multiplicity classes.


342 Signal extractions :
343 To estimate the systematic uncertainties due to signal extractions, we have varied combinatorial back-
344 ground normalization range, residual background function, fitting range, width of the distribution and
345 signal counting. The details of these variations is given below :
346
347 1. Combinatorial ackground normalization (Enorm) : The different normalization regions are se-
348 lected for K∗ and φ. In case of K∗, the background histogram is normalized in the region of invariant
349 mass between 1.1 to 1.15 GeV/c2 as a default range. For the systematic study the variation in the nor-
350 malization range on the right hand side of the signal is 1.0, 1.1, 1.1-1.2 GeV/c2 and on left hand side
351 of the signal 0.7-0.8 GeV/c2. In case of φ, the default range is 1.045 to 1.05 GeV/c2 and the systematic
352 variation is 0.995-1.0, 1.035-1.045 GeV/c2 .
353 2. Residual background function (Erb) : The default function used for the estimation of residual
Multiplicity dependence of resonances production 23
354 background if 2nd order polynomial for both K∗ and φ. The systematic uncertainty due residual back-
355 ground is estimated by comparing the 3rd order polynomial residual background with the default one.
356 3. Fitting range (E f it r ) : For φ -meson, fitting ranges are varied by 0.005 from the default fitting range
357 (DFR) i.e. [(DFR - 0.005, DFR + 0.005), (DFR + 0.005, DFR - 0.005)]. In case of K∗ the default range is
358 0.78 - 1.05 and for systematic study the range variations are 0.77-1.03, 0.76-1.06, 0.79-1.04.
359 4. Free width (Efw) : For φ, the width is fixed to the PDG value and extracted the resolution as
360 free parameter from the data in default settings.. Then we fixed these pT dependent resolutions for
361 systematic check. In case of K∗ width is kept free for systematic variation.
362 5. Signal count (Esc) : We have used method to calculate the signal : bin counting and function
363 integration. In case of K∗, bin counting method is used for systematic variation, however for φ function
364 integration is used for systematic variations. The systematic uncertainty due signal count is estimated
365 by comparing the default method with the other method.


366 Material budget (Emb) : The material budget systematic uncertainty is estimated by using the pT
367 dependent material budget uncertainty of single K and π. In simulation (official PYTHIA), for a given
368 pT , first the decay daughters of K∗ and φ mesons are identified. Then the daughters are assigned an
369 uncertainty corresponding to that pT bin. These uncertainties are linearly added (assigned for that of
370 K∗ or φ) and a TProfile is filled for each pT bin of K∗ and φ-meson. The bin content of the TProfile
371 which gives the average value is taken as the systematic uncertainty.


372 Hadronic interaction (Ehi ) :
373 The same procedure as described in case of material budget is followed to estimate the systematic
374 uncertainty due to the hadronic interaction for K∗ and φ -meson.
375 Total Systematic Uncertainty (Et ot al ) The total systematic uncertainty is the quadrature sum of all
376 sources.
E =􏰃E2 +E2 +E2 +E2 +E2 +E2 +E2 +E2 +E2 +E2 (22) tot sc rb fw ftr sc anacut pid gtrk mb hd
377 For the systematic study we repeat the measurement by varying one parameter at a time. A Barlow [6]
378 check has been performed for each measurement to verify whether it is due to a systematic effect or a
379 statistical fluctuation.



381 Barlow check
382 Let each measurement be indicated by (yi ± σi) and the ficentral valuefi (default measurement) by
383 (yc ±σc), one can define ∆σi (eqn. 23).
∆σi =􏰃|σi2−σc2| (23)
384 Then we calculate ni = ∆yi/∆σi, where ∆yi = |yc −yi|. If ni ≤ 1.0 then the effects are due to the statistical
385 fluctuation otherwise there is a systematic effect which means that the two measurements are not
386 compatible within the statistical errors.
387 The measurements which passed the fiBarlowfi check (ni > 1) are used to determine the systematic
388 uncertainty. For measurements N > 2, the systematic uncertainty has been determined as the RMS
389 (eqn. 24) of the available measurements otherwise the same is the absolute difference between them.
   
24 ALICE Analysis Note 2015
 􏰅12
δySyst. = N ∑(yi −y ̄) (24)
i
390 Here N is the total number of available measurements including yc and y ̄ is the average of value of the
391 measurements. Whenever only two measurements were available and did not pass “Barlow” check,
392 zero systematic uncertainty has been assigned to the value.

\end{comment}